\documentclass[11pt]{article}
\usepackage{a4, fullpage}
\usepackage{bibtopic}
\usepackage[small,compact]{titlesec}
\usepackage{float}
\usepackage{amssymb,amsmath}
\usepackage[T1]{fontenc}
\usepackage{graphicx}
\usepackage{multicol}
\restylefloat{table}
%\usepackage{parskip}
%\usepackage{setspace}




\setlength{\parskip}{0.3cm}
\setlength{\parindent}{0cm}
\setlength{\textheight}{10in}
\setlength{\textwidth}{6.5in}
\setlength{\parskip}{2pt}
\addtolength{\oddsidemargin}{-.3in}
\addtolength{\evensidemargin}{-.3in}
\addtolength{\topmargin}{-.6in}
\addtolength{\textwidth}{.6in}

\begin{document}



\title{Assignment 4 \\ Group 30  }

\author{John Walker \and Adam Fiksen \and Giovanni Charles }

\date{\today}         % inserts today's date

\maketitle           % generates the title from the data above



\section{Results}

\subsection{Comparison of values of F1 measure}

\begin{tabular}{c c c} % centered columns (4 columns)
\hline\hline %inserts double horizontal lines
Emotion & name & Decision Tree & Neural Network & CBR\\ [0.5ex] % inserts table
\hline % inserts single horizontal line
1 & anger     & 0 & 0 & 0  \\ % inserting body of the table
2 & disgust   & 0 & 0 & 0  \\
3 & fear      & 0 & 0 & 0  \\
4 & happiness & 0 & 0 & 0  \\
5 & sadness   & 0 & 0 & 0  \\ 
6 & suprise   & 0 & 0 & 0  \\ [1ex] % [1ex] adds vertical space
\hline %inserts single line
\end{tabular}

\subsection{T-test of the clean data}
\subsection{T-test of the noisy data}

\section{Questions}

\subsection{Which algorithm performed better overall in terms of values of F1 measure (part I)? Which algorithm performed better when comparison was performed using the t-test (part II and part III)? Can we claim that this algorithm is a better learning algorithm than the others in general? Why? Why not?}

\subsection{How did you adjust the significance level in order to take into account the fact that you perform a multiple comparison test?}
As we are performing multiple statistical tests this increases the chance that a significant result is produced simply by chance for example if you did 100 tests at the P < 0.1 significance level then you expect 10 of the 
tests to come back as significant simply by chance. We therefore had to adjust our signficance level to counter this problem, to do this we used the Bonferroni correction. The Bonferroni correction works by simply dividing
your significance (in our case 0.05) by the number of tests (in our case 3) to produce a new significance level for an individual test of 5 / 3; 

\subsection{Which type of t-test did you use and why?}

\subsection{Why do you think t-test was performed on the classification error and not the F1 measure? Whats the theoretical justification for this decision?}

\subsection{What is the trade-off between the number of folds you use and the number of examples per fold? In other words, what is going to happen if you use more folds, so you will have fewer examples per fold, or if you use fewer folds, so you will have more examples per fold?}

\subsection{Suppose that we want to add some new emotions to the existing dataset. Which of the examined algorithms are more suitable for incorporating the new classes in terms of engineering effort? Which algorithms need to undergo radical changes in order to include new classes?}




\end{document}
