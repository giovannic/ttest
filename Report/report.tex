\documentclass[11pt]{article}
\usepackage{a4, fullpage}
\usepackage{bibtopic}
\usepackage[small,compact]{titlesec}
\usepackage{float}
\usepackage{amssymb,amsmath}
\usepackage[T1]{fontenc}
\usepackage{graphicx}
\usepackage{multicol}
\restylefloat{table}
%\usepackage{parskip}
%\usepackage{setspace}




\setlength{\parskip}{0.3cm}
\setlength{\parindent}{0cm}
\setlength{\textheight}{10in}
\setlength{\textwidth}{6.5in}
\setlength{\parskip}{2pt}
\addtolength{\oddsidemargin}{-.3in}
\addtolength{\evensidemargin}{-.3in}
\addtolength{\topmargin}{-.6in}
\addtolength{\textwidth}{.6in}

\begin{document}



\title{Assignment 4 \\ Group 30  }

\author{John Walker \and Adam Fiksen \and Giovanni Charles }

\date{\today}         % inserts today's date

\maketitle           % generates the title from the data above



\section{Results}

\section{Questions}
\subsection{Which algorithm performed better overall in terms of values of F1 measure (part I)? Which algorithm performed better when comparison was performed using the t-test (part II and part III)? Can we claim that this algorithm is a better learning algorithm than the others in general? Why? Why not?}

\subsection{How did you adjust the significance level in order to take into account the fact that you perform a multiple comparison test?}

\subsection{Which type of t-test did you use and why?}

We used the inbuilt "ttest", paired t-test. We decided to use this since they are not independent between the different algorithms. Since each of our folds are decided deterministically our algorithms use the same training examples to predict corresponding labels so it. Therefore it makes sense to pair these predictions of the same data and compare the difference and check if they are similar.  

\subsection{Why do you think t-test was performed on the classification error and not the F1 measure? What‟s the theoretical justification for this decision?}

Each fold will have a different number of positive and negative examples. The F1 will not be identically distributed since it is an average of the recall and precision rates, a precise algorithm will perform better in folds with more positive examples skewing the results. 

This means the sample error for that fold is no longer an i.i.d. To calculate the variance of the sample errors the central limit theorm is used under the assumption that the samples are i.i.d, this means that the variance used in the t-test calculation will not be representative.

\subsection{What is the trade-off between the number of folds you use and the number of examples per fold? In other words, what is going to happen if you use more folds, so you will have fewer examples per fold, or if you use fewer folds, so you will have more examples per fold?}

Using smaller folds trains your algorithm using a larger variety of data and the average result over the larger sample of performance measures will produce more reliable results.

As a minor point, many small folds will cause a large overhead in computation and is not suggested for frequent evaluation of the performance. 

At the extremely small fold sizes the algorithms will experience underfitting to the data. The algorithm will not 'learn' enough from the fold to predict the rest of the data so will appear to perform worse than normal. Larger folds will fit closer to the data and will be more representitive of how the algorithm will perform with larger datasets. 

\subsection{Suppose that we want to add some new emotions to the existing dataset. Which of the examined algorithms are more suitable for incorporating the new classes in terms of engineering effort? Which algorithms need to undergo radical changes in order to include new classes?}

The eager learning algorithms, decision trees and neural networks, will require the most effort since they will have to be retrained from scratch using a dataset including examples of the new emotion. 

Of these neural networks will require the most computational effort due to the large number of parameters. The requirements to correctly predict the new emotion could move any number of parameters for network training meaning the search for an optimal network will have to be redone.

In the lazy learning, case based reasoning, algorithm it is possible for you to create a new cluster and feed it examples through the 'retain' method. However, an active system may have more of an understanding of the typicality of the current cases in the system and therefore the new emotion will be at a slight disadvantage and in the wrong environment it may be isolated and not perform as well. It is still possible to retrain the system from scratch to acheive better classification rates.
\end{document}
