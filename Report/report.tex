\documentclass[11pt]{article}
\usepackage{a4, fullpage}
\usepackage{bibtopic}
\usepackage[small,compact]{titlesec}
\usepackage{float}
\usepackage{amssymb,amsmath}
\usepackage[T1]{fontenc}
\usepackage{graphicx}
\usepackage{multicol}
\restylefloat{table}
%\usepackage{parskip}
%\usepackage{setspace}




\setlength{\parskip}{0.3cm}
\setlength{\parindent}{0cm}
\setlength{\textheight}{10in}
\setlength{\textwidth}{6.5in}
\setlength{\parskip}{2pt}
\addtolength{\oddsidemargin}{-.3in}
\addtolength{\evensidemargin}{-.3in}
\addtolength{\topmargin}{-.6in}
\addtolength{\textwidth}{.6in}

\begin{document}



\title{Assignment 4 \\ Group 30  }

\author{John Walker \and Adam Fiksen \and Giovanni Charles }

\date{\today}         % inserts today's date

\maketitle           % generates the title from the data above



\section{Results}

\section{Questions}
\subsection{Which algorithm performed better overall in terms of values of F1 measure (part I)? Which algorithm performed better when comparison was performed using the t-test (part II and part III)? Can we claim that this algorithm is a better learning algorithm than the others in general? Why? Why not?}

\subsection{How did you adjust the significance level in order to take into account the fact that you perform a multiple comparison test?}

\subsection{Which type of t-test did you use and why?}

We used the inbuilt ttest2

\subsection{Why do you think t-test was performed on the classification error and not the F1 measure? What‟s the theoretical justification for this decision?}

The provided target values have the following percentage of the 6 emotions:

\begin{tabular}{c c c} % centered columns (4 columns)
\hline\hline %inserts double horizontal lines
Emotion & name & Percentage\\ [0.5ex] % inserts table
\hline % inserts single horizontal line
1 & anger     & 13 \\ % inserting body of the table
2 & disgust   & 19 \\
3 & fear      & 12 \\
4 & happiness & 21 \\
5 & sadness   & 13 \\ 
6 & suprise   & 20 \\ [1ex] % [1ex] adds vertical space
\hline %inserts single line
\end{tabular}

This means that the provided targets are unbalanced, there is not an equal chance of getting each emotion. Using the classification function as a fitness function would be misleading since it is not weighted for each emotion.

Using the precision rate would weight the true positives against the positive examples of the emotion and the recall rate will weight the false negative against the negative examples. The average of this will give our f1 measure which is more representative of our performance. 

\subsection{What is the trade-off between the number of folds you use and the number of examples per fold? In other words, what is going to happen if you use more folds, so you will have fewer examples per fold, or if you use fewer folds, so you will have more examples per fold?}

Using smaller folds trains your algorithm using a larger variety of data and the average result over the larger sample of performance measures will produce more reliable results.

As a minor point, many small folds will cause a large overhead in computation and is not suggested for frequent evaluation of the performance. 

At the extremely small fold sizes the algorithms will experience underfitting to the data. The algorithm will not 'learn' enough from the fold to predict the rest of the data so will appear to perform worse than normal. Larger folds will fit closer to the data and will be more representitive of how the algorithm will perform with larger datasets. 

\subsection{Suppose that we want to add some new emotions to the existing dataset. Which of the examined algorithms are more suitable for incorporating the new classes in terms of engineering effort? Which algorithms need to undergo radical changes in order to include new classes?}


\end{document}
